\section{Usage}
In order to prevent name conflicts and keep global scope clean everything is placed into
namespace \codet{owon}. In the below code the \codet{using} declaration is \emph{not}
assumed so each entity of the library appears with the \codet{owon::} prefix.
However, it would be useful to make a namespace alias to type less (see below). 

This chapter covers two forms of usage:
\begin{itemize}
    \item as a stand-alone pure C++ software
    \item as a part of the software based on the \textbf{ROOT CERN} framework
\end{itemize}

\subsection{Stand-alone}
This software is used to get two kinds of information about a waveform: \emph{primary}
and \emph{secondary}. Primary information consists of data encoded in a binary file itself
such as \emph{voltage per division, number of channels, data points, 
etc}.
Secondary information contains results of some analysis of data points:
\emph{baseline level,
amplitude, integral, etc} and, obviously, based on the primary one.

\subsubsection*{Terminology} Extracting primary and secondary information from a binary
file is called \emph{parsing} and \emph{analysis}, respectively, in this document.

\subsubsection{Parsing}\label{sssec:parsing}
To parse a binary file the following algorithm should be executed:
\begin{enumerate}
    \item Choose a model namespace
    \item Create an instance of the \codet{parser} class
    \item Call \codet{parser::parse} member function
    \item Choose the channel with \codet{parser::set\tus active\tus channel} member function
    \item Get information via get-methods
    \captionof{algorithm}{Parsing of a binary file}
    \label{algo:parse}
\end{enumerate}
Below each step of the above algorithm is explained in detail.

\paragraph*{Model namespace.}
As it was mentioned above everything is contained in the \codet{owon} namespace.
The \codet{owon} namespace itself includes separate namespace for each model with
a name the same as the model name (lowercase). Such a namespace will be called
\emph{a model namespace} in the below text. To choose model namespace corresponding
to your scope you include its header which is
\begin{lstlisting}
device/<model_name>.h
\end{lstlisting}
where \codet{<model\tus name>} matches the model name with the letters in lower case.
Also it is convenient to create a namespace alias to type less. For example, if your
oscilloscope is SDS6062 then the following two lines should be in your code (though the second one is not mandatory):
\begin{lstlisting}
#include "device/sds6062.h"

namespace osc = owon::sds6062;//alias
\end{lstlisting}

\paragraph*{The \codet{parser} class.}A model namespace contains the \codet{parser} class --- the class that is responsible
for the extraction information (parsing) from binary files. The \codet{parser} class is
one of the two (along with the \codet{analyzer} class) most important tools in this
software. At least one instance of this class must be created:
\begin{lstlisting}
osc::parser p;//create parser
\end{lstlisting}

\paragraph{The \codet{parser::parse} member.} Once an instance of the \codet{parser} class
is created you use its member function \codet{parse} to extract information from a binary.
It takes the only argument --- path to the binary file (it may be either absolute or relative,
with an extension) to be parsed:
\begin{lstlisting}
p.parse( "path/to/binary/file.bin" );//NOTE: the extension is present
\end{lstlisting}
\paragraph{Setting the active channel and getting information.}
After successful (see below) execution of the above
code the extracted information is
accessible through the get-methods of the \codet{parser} class. However, most of these
methods are intended to provide information about \emph{individual} channel:
\emph{voltage per division, time per division, data points, etc}. This channel is
called the \emph{active channel} in the software.
Use \codet{parser::set\tus active\tus channel} member function to choose what channel is active:

\begin{lstlisting}
p.set_active_channel( owon::CH1 );//select the channel
//returns voltage per division of the active channel --- CH1
p.get_voltage_per_div();

p.set_active_channel( owon::CH2 );//select the channel
//returns voltage per division of the active channel --- CH2
p.get_voltage_per_div();

//iterate over data points of the active channel --- CH2
for( sds::parser::const_point_iterator point = p.cbegin(); point != p.cend(); ++point )
{
    //do something with points...
    std::cout << "Point is ( " << point->time << ", " << point->voltage << ") \n";
}
\end{lstlisting}
For the full list of availalble get-methods see Sec.~\ref{ssec:ref:osc}~(\codet{class oscilloscope}).

\subsubsection{Analysis}\label{sssec:anal}
The \codet{owon} namespace also includes the \codet{analyzer} class. It performs simple
but often very useful operations on a waveform: calculation of the baseline level, finding
the point of maximum amplitude, etc. Use the following algorithm to analyse a waveform: 

\begin{enumerate}
    \item Create an instance of the \codet{parser} class (see Sec.~\ref{sssec:parsing})
    \item Parse a waveform (see Sec.~\ref{sssec:parsing})        
    \item Create an instance of the \codet{analyzer} class 
    \item Set the channel to be analyzed as the active one (see Sec.~\ref{sssec:parsing})
    \item Set the analysis config if needed 
    \item Call \codet{analyzer::analyze} member function to perform analysis 
    \item Call get-methods of the \codet{analyzer} class to get the results
    \captionof{algorithm}{Waveform analysis}
    \label{algo:anal}
\end{enumerate}
This class is not autonomous --- it works in pair with the \codet{parser} class. Before
instantiating the \codet{analyzer} an instance of the \codet{parser} class must have been
created. Therefore, the first two steps are required.

\begin{lstlisting}
#include "analyzer.h"
#include "device/sds6062.h"

namespace osc = owon::sds6062;

//somewhere in the code
osc::parser p;
p.parse( "path/to/binary/file.bin" )
...
\end{lstlisting}

\paragraph*{The \codet{analyzer} class.} The constructor of the \codet{analyzer} class
takes pointer to a \codet{parser} object as an argument:
\begin{lstlisting}
owon::analyzer a( &p );//create analyzer
\end{lstlisting}

\paragraph*{The active channel.} The role of the active channel (see Sec.~\ref{sssec:parsing})
in the analysis process is simple --- all the analysis is performed on the data points
of the active channel.

\paragraph*{The analysis config.} Next step is to set the analysis config

\begin{lstlisting}
a.set_baseline_time( p.get_trigger_time() );//set the range for baseline calculation from the beginning of the waveform to its trigger time
a.set_gate( 0.3, 0.8 );//set the integration range from 300 to 800 ns

a.analyze();
\end{lstlisting}

\paragraph*{Getting the results.} Use the get-methods of the \codet{analyzer} class to
get the results of analysis:
\begin{lstlisting}
a.get_baseline();
a.get_integral();
\end{lstlisting}
For the full list of the get-methods see Sec.~\ref{ssec:ref:anal} (\codet{class analyzer}).

\newpage
\subsubsection{Compilation}
Here is the code of a minimal example and how to compile it.
\begin{lstlisting}[numbers=left]
#include <iostream>

#include "device/sds6062.h"
#include "analyzer.h"

namespace osc = owon::sds6062;//alias


int main()
{
    osc::parser p;//create parser
    owon::analyzer a( &p );//create analyzer
    
    p.parse( "path/to/binary/file.bin" );//NOTE: the extension is present
    
    if( p.get_status() )//if successfully parsed
    {
        p.print();
    
        a.set_gate( -1, -1 );//see reference guide
        a.set_baseline_time( -1 );//see reference guide
    
        a.analyze();
        a.print();
    }
    else
    {
        std::cerr << "Something went wrong" << std::endl;
    }
    
    return 0;
}
    
\end{lstlisting}
Provided you have saved the above code to the file named \codet{example.cpp}, and
appended the \codet{CPLUS\tus INCLUDE\tus PATH} environment variable as it was
recommended in Sec.~\ref{sec:install} the compilation is as follows:
\begin{lstlisting}[language=bash]
g++ example.cpp -std=c++11 -lowonparse -o example
\end{lstlisting}
However, it is more convenient to use \codet{Makefile} to compile your code.
You could find examples of simple \codet{Makefile}s in the
\codet{<OWON\tus DIR>/examples} directory.

\subsection{ROOT-compatible}\label{ssec:ROOT}
In this section it will be explained how to use this library together with the ROOT CERN
framework, in particular, how to create a \codet{TTree} from several binary files. Provided
you have followed the instructions in Sec.~\ref{ssec:install:ROOT} the following algorithm
should be used to create a \codet{TTree} from a set of a binary files:
\begin{enumerate}
    \item Create an instance of the \codet{parser} class in the model namespace
        corresponding to the model of an oscilloscope that was used to record waveforms
    \item Create an instance of the \codet{TreeCreator} class using previously created
        \codet{parser}
    \item Set analysis config and the active channel using member-functions of the \codet{TreeCreator} class
    \item Call the \codet{CreateTree} member-function
    \item Compile and run
    \label{algo:tree}
    \captionof{algorithm}{Creation of a \codet{TTree}}
\end{enumerate}
Below is the explanation the above algorithm in detail. 

\paragraph*{The \codet{TreeCreator} class.} This class is responsible for (no surprise) 
creation of a \codet{TTree} from 
To create an instance of this class you, firstly, include its header:
\begin{lstlisting}
#include "ROOT/TreeCreator.h"
\end{lstlisting}
Then you use the only constructor of this class which takes four arguments:
\begin{lstlisting}
TreeCreator( owon::oscilloscope* osc,
               const std::string& pathToDataDir,
               const std::string& pathToTreeFile,
               const std::string& treeFileName )
\end{lstlisting}
where
    \begin{itemize}
        \item[] \codet{osc} --- pointer to the \codet{oscilloscope} class (which the \codet{parser} class in each model namespace inherits from) that is of the same model that is
            used to record the binaries
        \item[] \codet{pathToDataDir} --- path to the directory containing the binaries (see the NOTE below)
        \item[] \codet{pathToTreeFile} --- path to the location where the resulting \codet{TTree} will be placed
        \item[] \codet{treeFileName} --- name of a \codet{.root} file (without an extension) with the resulting \codet{TTree}
    \end{itemize}
The meaning of the second argument requires additional explanation. There may be two
cases here. The first is the one when the target directory contains no other directories
(subdirectories) --- only binary files. I.e. the structure is the following:
\begin{lstlisting}[language=bash]
path
|
|_to
  |
  |_data <--target
     |
     |__file1.bin
     |__file2.bin
     |__...
\end{lstlisting}
In this case, provided the second argument in the \codet{TreeCreator} constructor
was \codet{"path/to/data"}, THE ONLY \codet{TTree} named \codet{tree\tus} would be constructed
from all the binaries in the \codet{data} directory. If, on the other hand, the target
directory contains other directories, for example:
\begin{lstlisting}[language=bash]
path
|
|_to
  |
  |_data <--target
    |
    |_data1
    |   | 
    |   |__file1.bin
    |   |__file2.bin
    |   |__...
    |
    |_data2
      |
      |__file1.bin
      |__file2.bin
      |__...
\end{lstlisting}
then, provided the second argument in the \codet{TreeCreator} constructor
was \codet{"path/to/data"}, TWO \codet{TTree}s named \codet{tree\tus data1} and
\codet{tree\tus data2} would be constructed from all the binaries in the \codet{data1} and
\codet{data2} directory (and in its subdirectories), respectively.
It means that hierarchy of the subdirectories of the target directory doesn't matter:
all the binaries are searched recursively in a subdirectory of the target directory. I.e.
the \codet{subsubdata} is expanded when constructing a \codet{TTree} provided the following
structure:
\begin{lstlisting}[language=bash]
path
|
|_to
  |
  |_data <--target
    |
    |_data1 <-- matters
    |   | 
    |   |__file1.bin
    |   |__file2.bin
    |   |__...
    |
    |_data2 <--matters
      |
      |__file1.bin
      |__file2.bin
      |__...
      |__subsubdata <-- DOESN'T matter: expanded
         |__file1.bin
         |__file2.bin
         |__...
\end{lstlisting}
However, the path (see below) of the every binary is conserved in the resulting
\codet{TTree}.

After the instantiating the \codet{TreeCreator} you should set the analysis config and
the active channel
(see Sec.~\ref{sssec:anal}) which will be used when constructing the \codet{TTree}:
\begin{lstlisting}
    tc.SetBaselineTime( 0 );//from the beginning to the trigger time
    tc.SetGate( 0, 3 );//from the beginning to 3 us after the trigger
    tc.SetActiveChannel( owon::CH1 );//the default one
\end{lstlisting}

\Note{Setting the analysis config for the \codet{TreeCreator} class differs slightly from that for the \codet{analyzer} class (see~Sec.\ref{ssec:ref:tree})}

After that you call the \codet{CreateTree} member-function. The whole code could be:

\begin{lstlisting}
#include "ROOT/TreeCreator.h"

#include "device/tds8204.h"


void CreateTree()
{
    owon::tds8204::parser p;//OWON TDS8204 was used for recording

    TreeCreator tc( &p, "./Data", "./", "myFirstTree" );
    tc.SetBaselineTime( 0 );
    tc.SetGate( 0, 3 );
    tc.SetActiveChannel( owon::CH1 );//the default one

    tc.CreateTree();
}
\end{lstlisting}

\paragraph*{Compile and run.} To run the above example create (or append) the \codet{rootlogon.C} file in the directory
where you put this example with the following line:
\begin{lstlisting}
{
    gSystem->AddLinkedLibs( "path/to/<OWON_DIR>/src/ROOT/TreeCreator/TreeCreator_cpp.so" );
}
\end{lstlisting}
where \codet{/path/to/<OWON\tus DIR>} is the absolute path to the \codet{<OWON\tus DIR>}.

Then type the following in your working directory (it is assumed you have saved the code in the file \codet{CreateTree.C}):
\begin{lstlisting}
root -l CreateTree.C+
\end{lstlisting}
After that there will appear the file (along with the others) named \codet{myFirstTree.root} in the current directory with a \codet{TTree}. The full example (with the demo data)
see in the \codet{<OWON\tus DIR>/example/TDS/ROOT/create\tus tree} directory.
