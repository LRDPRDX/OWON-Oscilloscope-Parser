\newpage
\section{Reference guide}\label{sec:ref}

This reference guide is not considered to be complete. In this section only \codet{public}
members are explained. This is rather a \emph{usage reference guide}.

\Note{The \codet{owon::} prefix is omitted in the below code (except the ROOT-compatible part)}

\subsection{namespace \codet{owon}}

\subsubsection*{Description}

\hspace{\parindent} The library is entirely contained in the \codet{owon} namespace (except the ROOT-compatible parts). The
schematics of the library structure is shown in Fig.~\ref{fig:diag}.

\tikzstyle{namespace}=[rectangle, draw=black, rounded corners, fill=white, drop shadow,
        text centered, anchor=north, text=black, text width=3cm]
\tikzstyle{class}=[rectangle, draw=black, rounded corners, fill=blue!20, drop shadow,
        text centered, anchor=north, text=black, text width=3cm]
\tikzstyle{struct}=[rectangle, draw=black, rounded corners, fill=green!20, drop shadow,
        text centered, anchor=north, text=white, text width=3cm]
\tikzstyle{derive}=[->, >=open triangle 90, thick]
\tikzstyle{aggregate}=[->, >=diamond, thick]
\tikzstyle{line}=[thick]
\tikzstyle{contain}=[-o]
        
\begin{figure}[H]
\centering
\begin{tikzpicture}[node distance=3cm]
    \node (owon) [namespace]
        {
            \codet{owon} 
        };
    \node (tds8204) [namespace, below=2cm of owon]
        {
            \codet{tds8204}
        };
    \node (sds6062) [namespace, left=2cm of tds8204]
        {
            \codet{sds6062}
        };
    \node (sds7072) [namespace, right=2cm of tds8204]
        {
            \codet{sds7072}
        };
    \node (analyzer) [class, rectangle split, rectangle split parts=2, above=2cm of owon]
        {
            \codet{analyzer}
            \nodepart{second}
                \begin{itemize}
                    \item[+] \codet{analyze()}
                    \item[+] \codet{set\tus gate()}
                \end{itemize}
        };
    \node (oscilloscope) [class, rectangle split, text width=6cm,  rectangle split parts=3, left=1cm of analyzer]
        {
            \codet{oscilloscope}
            \nodepart{second}
                \begin{itemize}
                    \item[+] \codet{vector<channel> channels}
                \end{itemize}
            \nodepart{third}
                \begin{itemize}
                    \item[+] \codet{virtual parse() = 0}                 
                    \item[+] \codet{get\tus wave\tus size()}
                    \item[+] \codet{set\tus active\tus channel()}
                \end{itemize}
        };
    \node (parser) [class, rectangle split, text width=3cm, rectangle split parts=2, below=1cm of tds8204]
        {
            \codet{parser} 
            \nodepart{second}
                \begin{itemize}
                    \item[+] \codet{parse()}
                \end{itemize}
        };
    \node (parser2) [class, rectangle split, text width=3cm, rectangle split parts=2, below=1cm of sds6062]
        {
            \codet{parser} 
            \nodepart{second}
                \begin{itemize}
                    \item[+] \codet{parse()}
                \end{itemize}
        };
    \node (parser3) [class, rectangle split, text width=3cm, rectangle split parts=2, below=1cm of sds7072]
        {
            \codet{parser} 
            \nodepart{second}
                \begin{itemize}
                    \item[+] \codet{parse()}
                \end{itemize}
        };

    \node (channel) [class, rectangle split, text width=5cm, rectangle split parts=2, right=1cm of analyzer]
        {
            \codet{channel} 
            \nodepart{second}
                \begin{itemize}
                    \item[+] \codet{vector<point> points}                 
                \end{itemize}
        };

    \node (point) [struct, rectangle split, text width=3cm, rectangle split parts=2, right=2cm of owon]
        {
            \codet{point} 
            \nodepart{second}
                \begin{itemize}
                    \item[+] \codet{time}
                    \item[+] \codet{voltage}
                \end{itemize}
        };

        \draw[contain] (owon.south) to (tds8204.north);
        \draw[contain] (owon.south) to (sds6062.north);
        \draw[contain] (owon.south) to (sds7072.north);
        \draw[contain] (owon.north) to (oscilloscope.south);
        \draw[contain] (owon.north) to node[right]{contains} (analyzer.south);
        \draw[contain] (sds6062.south) to (parser2.north);
        \draw[contain] (sds7072.south) to (parser3.north);
        \draw[contain] (tds8204.south) to (parser.north);
        \draw[contain] (owon.east) to (point.west);

        \draw[aggregate] (point.north) to (channel.south);

        \node (aux) [left=1cm of oscilloscope]{};
        \draw[line] (parser2.west) -| (aux.east);
        \draw[derive] (aux.east) to (oscilloscope.west);

        \node (aux2) [above=1cm of oscilloscope]{};
        \draw[line] (channel.north) |- (aux2.south);
        \draw[aggregate] (aux2.south) to (oscilloscope.north);
\end{tikzpicture}
\caption{Structure of the library. Colorless rectangles --- \codet{namespace}s, green --- \codet{structure}s, blue --- \codet{classe}s. Open triangle arrows --- inheritance, diamonds --- aggregation, circles --- containing}
\label{fig:diag}
\end{figure}


\subsubsection*{\codet{enum}s}

%OWON :: enums
\begin{lstlisting}
enum CHANNEL : unsigned { CH1, CH2, CH3, CH4 };
\end{lstlisting}

\begin{tabularx}{0.9\textwidth}{rp{11cm}}
\toprule
Description: & Represents the list of available channels\\
Values: & \codet{CH1, CH2, CH3, CH4}\\
\bottomrule
\end{tabularx}
\vspace{1cm}

%OWON :: members
\subsubsection*{Functions}

%OWON :: ch_to_str
\begin{lstlisting}
inline std::string ch_to_str( const CHANNEL ch );
\end{lstlisting}

\begin{tabularx}{0.9\textwidth}{rp{11cm}}
\toprule
    Description: & Represents the list of available channels\\
    Arguments:   &
        \begin{tabular}[t]{@{\hspace{0em}}l@{}@{\hspace{1em}}l@{}l}
            \codet{ch} - & channel value\\
        \end{tabular}\\
    Return:      & string representation of the channel\\
\bottomrule
\end{tabularx}
\vspace{1cm}

%OWON :: << ch
\begin{lstlisting}
inline std::ostream& operator<<( std::ostream& os, const CHANNEL& ch )
\end{lstlisting}

\begin{tabularx}{0.9\textwidth}{rp{11cm}}
\toprule
    Description:    & Allows user to use stream insertion operator \codet{<<} to print channel values\\
    Usage example:  & \codet{std::cout << owon::CH2 << std::endl;}\\
\bottomrule
\end{tabularx}
\vspace{1cm}

%OWON :: classes
\subsubsection*{\codet{strust}s and \codet{classe}s}

%OWON :: point
\begin{lstlisting}
struct point
\end{lstlisting}

\begin{tabularx}{0.9\textwidth}{rp{11cm}}
\toprule
    Description:    & Represents a single data point. Each channel (\codet{channel} class) contains a vector of points (\codet{std::vector<point>}) and each oscilloscope (\codet{oscilloscope} class) has an iterator (\codet{const\tus point\tus iterator}) that iterates over
    the points of its active channel\\
    Fields:   &
        \begin{tabular}[t]{@{\hspace{0em}}l@{}@{\hspace{1em}}l@{}l}
            \codet{float time} & - time mark of a data point (in $\mu$s)\\
            \codet{float voltage} & - voltage mark of a data point (in mV)\\
        \end{tabular}\\
\bottomrule
\end{tabularx}
\vspace{1cm}

%OWON :: channel
\begin{lstlisting}
class channel
\end{lstlisting}

\begin{tabularx}{0.9\textwidth}{rp{11cm}}
\toprule
    Description:    & Represents an individual channel. User doesn't use this class directly\\
\bottomrule
\end{tabularx}
\vspace{1cm}

%OWON :: oscilloscope
\begin{lstlisting}
class oscilloscope
\end{lstlisting}

\begin{tabularx}{0.9\textwidth}{rp{11cm}}
\toprule
    Description:    & Abstract. Base class for all parsers. The \codet{parser} class in every model namespace (see Sec.~\ref{sssec:parsing}) derives this class. It has all the get-methods to retrieve the parsed information.\\
\bottomrule
\end{tabularx}
\vspace{1cm}

%OWON :: analyzer
\begin{lstlisting}
class analyzer
\end{lstlisting}

\begin{tabularx}{0.9\textwidth}{rp{11cm}}
\toprule
    Description:    & Performs some analysis of a waveform\\
\bottomrule
\end{tabularx}
\vspace{1cm}

\newpage
\subsection{class \codet{oscilloscope}}\label{ssec:ref:osc}
\subsubsection*{Description}

\hspace{\parindent} --- 

\subsubsection*{\codet{public typedef}s}
\begin{lstlisting}
typedef typename std::vector<owon::point>::const_iterator const_point_iterator;  
\end{lstlisting}
\begin{tabularx}{0.9\textwidth}{rp{11cm}}
\toprule
    Description: & Should be used to iterate over the data points in a waveform\\
\bottomrule
\end{tabularx}
\vspace{1cm}

\subsubsection*{\codet{public} members}
\begin{lstlisting}
const_point_iterator cbegin() const noexcept
\end{lstlisting}
\begin{tabularx}{\textwidth}{rp{11cm}}
    \toprule
    Description: & Should be used to get the first point of a waveform (\textcolor{red}{AC})\\
    Return: & Iterator to the beginning\\
    Arguments: & None\\
    \bottomrule
\end{tabularx}
\vspace{1cm}

\begin{lstlisting}
const_point_iterator cend() const noexcept
\end{lstlisting}
\begin{tabularx}{\textwidth}{rp{11cm}}
    \toprule
    Description: & Should be used to check if the end of a waveform is reached (\textcolor{red}{AC})\\
    Return: & Iterator to the end\\
    Arguments: & None\\
    \bottomrule
\end{tabularx}
\vspace{1cm}

\begin{lstlisting}
bool set_active_channel( CHANNEL active_channel )
\end{lstlisting}
\begin{tabularx}{\textwidth}{rp{11cm}}
    \toprule
    Description: & All the information that is of an individual channel (such as number of data points or voltage per devision) is returned for so called active channel (see Sec.\ref{sssec:parsing}). Therefore, in general getting the information has two stages: setting the
    active channel and then using the get-methods. NOTE: setting the active channel may fail in case when user tries to set unavailable (not existing) channel as the active one. For example, setting the channel \codet{CH3} as the active one on the \codet{sds6062} model is
    meaningless because that model has only two channels. In such a case the active channel defaults to \codet{CH1}\\
    Return: & \codet{true} if channel setting succeeded. \codet{false} otherwise\\
    Arguments:   &
        \begin{tabular}[t]{@{\hspace{0em}}l@{}@{\hspace{1em}}l@{}l}
            \codet{CHANNEL active\tus channel} & - channel to be the active one\\
        \end{tabular}\\
    \bottomrule
\end{tabularx}
\vspace{1cm}

\begin{lstlisting}
std::string get_serial() const
\end{lstlisting}
\begin{tabularx}{\textwidth}{rp{11cm}}
    \toprule
    Description: & Should be used to get the serial number of the oscilloscope that produced a given binary file\\
    Return: & Serial number string. It returns \codet{"N/A"} in case when serial number is not available\\
    Arguments: & None\\
    \bottomrule
\end{tabularx}
\vspace{1cm}

\begin{lstlisting}
std::string get_model() const
\end{lstlisting}
\begin{tabularx}{\textwidth}{rp{11cm}}
    \toprule
    Description: & Should be used to get the model name of the oscilloscope of a given parser (in model namespace)\\
    Return: & Model name string\\
    Arguments: & None\\
    \bottomrule
\end{tabularx}
\vspace{1cm}

\begin{lstlisting}
bool        get_status() const
\end{lstlisting}
\begin{tabularx}{\textwidth}{rp{11cm}}
    \toprule
    Description: & Returns the current status of the last performed parsing\\
    Return: & \codet{true} if the last parsing succeeded. \codet{false} otherwise\\
    Arguments: & None\\
    \bottomrule
\end{tabularx}
\vspace{1cm}

\begin{lstlisting}
unsigned    get_n_channels() const
\end{lstlisting}
\begin{tabularx}{\textwidth}{rp{11cm}}
    \toprule
    Description: & Returns the number of available channel of a given oscilloscope\\
    Return: & Number of available channels\\
    Arguments: & None\\
    \bottomrule
\end{tabularx}
\vspace{1cm}

\begin{lstlisting}
CHANNEL     get_active_channel() const
\end{lstlisting}
\begin{tabularx}{\textwidth}{rp{11cm}}
    \toprule
    Description: & Should be used to monitor the current active channel\\
    Return: & Current active channel\\
    Arguments: & None\\
    \bottomrule
\end{tabularx}
\vspace{1cm}

\begin{lstlisting}
unsigned    get_wave_size() const
\end{lstlisting}
\begin{tabularx}{\textwidth}{rp{11cm}}
    \toprule
    Description: & Should be used to know how many points in a waveform (\textcolor{red}{AC})\\
    Return: & Number of data points\\
    Arguments: & None\\
    \bottomrule
\end{tabularx}
\vspace{1cm}

\begin{lstlisting}
float       get_voltage_per_div() const
\end{lstlisting}
\begin{tabularx}{\textwidth}{rp{11cm}}
    \toprule
    Description: & Should be used to get voltage per division (\textcolor{red}{AC})\\
    Return: & Voltage per division (in mV)\\
    Arguments: & None\\
    \bottomrule
\end{tabularx}
\vspace{1cm}

\begin{lstlisting}
float       get_vertical_position() const
\end{lstlisting}
\begin{tabularx}{\textwidth}{rp{11cm}}
    \toprule
    Description: & Should be used to get vertical position of a zero level (vertical shift). NOTE: it is not necessary the position of the true ground (\textcolor{red}{AC})\\
    Return: & Vertical position (in mV)\\
    Arguments: & None\\
    \bottomrule
\end{tabularx}
\vspace{1cm}

\begin{lstlisting}
float       get_time_per_div() const
\end{lstlisting}
\begin{tabularx}{\textwidth}{rp{11cm}}
    \toprule
    Description: & Should be used to get time per division (\textcolor{red}{AC})\\
    Return: & Time per division (in $\mu$s)\\
    Arguments: & None\\
    \bottomrule
\end{tabularx}
\vspace{1cm}

\begin{lstlisting}
float       get_time_step() const
\end{lstlisting}
\begin{tabularx}{\textwidth}{rp{11cm}}
    \toprule
    Description: & Should be used to get the time between two consecutive points (\textcolor{red}{AC})\\
    Return: & Time between two consecutive points (in $\mu$s)\\
    Arguments: & None\\
    \bottomrule
\end{tabularx}
\vspace{1cm}

\begin{lstlisting}
float       get_trigger_time() const
\end{lstlisting}
\begin{tabularx}{\textwidth}{rp{11cm}}
    \toprule
    Description: & Returns the time of the trigger arrival counting from 0\\
    Return: & Trigger time (in $\mu$s)\\
    Arguments: & None\\
    \bottomrule
\end{tabularx}
\vspace{1cm}

\begin{lstlisting}
float       get_length() const
\end{lstlisting}
\begin{tabularx}{\textwidth}{rp{11cm}}
    \toprule
    Description: & Should be used to get the length of a waveform (\textcolor{red}{AC})\\
    Return: & Length of a signal (in $\mu$s)\\
    Arguments: & None\\
    \bottomrule
\end{tabularx}
\vspace{1cm}

\begin{lstlisting}
float       get_height() const
\end{lstlisting}
\begin{tabularx}{\textwidth}{rp{11cm}}
    \toprule
    Description: & Should be used to get the height of a waveform. It simply the visible voltage range on the oscilloscope window (\textcolor{red}{AC})\\
    Return: & Height of a signal (in mV)\\
    Arguments: & None\\
    \bottomrule
\end{tabularx}
\vspace{1cm}

\begin{lstlisting}
void print() const
\end{lstlisting}
\begin{tabularx}{\textwidth}{rp{11cm}}
    \toprule
    Description: & Prints the primary information of the content of a binary file\\
    Return: & None\\
    Arguments: & None\\
    \bottomrule
\end{tabularx}
\vspace{1cm}


\newpage
\subsection{class \codet{analyzer}}\label{ssec:ref:anal}
\subsubsection*{Description}

\hspace{\parindent} --- 

\subsubsection*{\codet{public enum}s}
\begin{lstlisting}
enum BASELINE { MODE, AVERAGE };
\end{lstlisting}
\begin{tabularx}{\textwidth}{rp{11cm}}
    \toprule
    Description: & Should be used to choose method to calculate the baseline level (see below)\\
    Values: & \codet{MODE, AVERAGE}\\
    \bottomrule
\end{tabularx}
\vspace{1cm}

\subsubsection*{\codet{public} members}
\begin{lstlisting}
analyzer( const oscilloscope* osc )
\end{lstlisting}
\begin{tabularx}{\textwidth}{rp{11cm}}
    \toprule
    Description: & Constructor\\
    Arguments: &
        \begin{tabular}[t]{@{\hspace{0em}}l@{}@{\hspace{1em}}l@{}l}
            \codet{const oscilloscope* osc} & - pointer to an \codet{oscilloscope} object\\
        \end{tabular}\\
    \bottomrule
\end{tabularx}
\vspace{1cm}

\begin{lstlisting}
void analyze()
\end{lstlisting}
\begin{tabularx}{\textwidth}{rp{11cm}}
    \toprule
    Description: & This function performs the analysis process. NOTE: before the invocation this function the \codet{oscilloscope::parse} function must have been called and the analysis config set (see Sec.~\ref{sssec:anal})\\
    Return: & None\\
    Arguments: & None\\
\end{tabularx}
\vspace{1cm}

\begin{lstlisting}
void set_baseline_time( float interval )
\end{lstlisting}
\begin{tabularx}{\textwidth}{rp{11cm}}
    \toprule
    Description: & Sets the range to be used during the baseline calculation (see below).
    If \codet{interval} $<0$ or exeeds the length of a waveform the range $[0, 0.5t_{\mathrm{trigger}}]$ will be used

    \Note{The \codet{parser::parse} must have been called before setting the baseline interval}\\
    Return: & None\\
    Arguments: &
        \begin{tabular}[t]{@{\hspace{0em}}l@{}@{\hspace{1em}}l@{}l}
            \codet{float interval} & - right edge of the range used to calculate\\
            & the baseline (left is zero) (in $\mu$s). Default is \codet{-1}\\
        \end{tabular}\\
    \bottomrule
\end{tabularx}
\vspace{1cm}

\begin{lstlisting}
void set_gate( float integral_start, float integral_stop )
\end{lstlisting}
\begin{tabularx}{\textwidth}{rp{11cm}}
    \toprule
    Description: & Should be used to specify the integration range. The integration processis simply the sum of the voltage displacements from the baseline level:
    \begin{equation}
        \mathrm{integral} = \delta T \sum_{t_i\in[t_{\mathrm{start}},t_{\mathrm{stop}}]} (v_0 - v_{i}),
    \end{equation}
    where
    \begin{itemize}
        \item[] $\delta T$ - time step between two consecutive points
        \item[] $v_0$ - the baseline level
        \item[] $v_i$ - voltage mark of the $i$-th point
        \item[] $t_i$ - time mark of the $i$-th point
        \item[] $t_{\mathrm{start}}, t_{\mathrm{stop}}$ - the beginning and the end of 
            integration, respectively
    \end{itemize}
    If \codet{interval\tus start} $<0$ or exceeds the length of a waveform it will be set to 0.
    If \codet{interval\tus stop} $<0$ or exceeds the length of a waveform it will be set to the length of a waveform
    
    \Note{The \codet{parser::parse} must have been called before setting the gate}\\
    Return: & None\\
    Arguments: &
        \begin{tabular}[t]{@{\hspace{0em}}l@{}@{\hspace{1em}}l@{}l}
            \codet{float integral\tus start} & - the time to itegrate from (in $\mu$s). Default is {-1}\\
            \codet{float integral\tus stop} & - the time to integrate to (in $\mu$s). Default is \codet{-1}\\
        \end{tabular}\\
    \bottomrule
\end{tabularx}
\vspace{1cm}

\begin{lstlisting}
void set_baseline_method( BASELINE method )
\end{lstlisting}
\begin{tabularx}{\textwidth}{rp{11cm}}
    \toprule
    Description: & Should be used to choose a method to calculate the baseline level.
    Two methods of the baseline calculation are available in the library. The first one is the mode-method. To
    choose it you should use the \codet{analyzer::BASELINE::MODE} as an argument. In this
    case the baseline is calculated simply as the mode of the voltages in the specified range:
    \begin{equation}
        \mathrm{baseline} = \underset{t_{i} \in [0, t_{\mathrm{baseline}}] }{\mathrm{mode}}\{ v_{i} \},
        \label{eq:ref:mode}
    \end{equation}
    where
    \begin{itemize}
        \item[] $v_{i}$ - voltage mark of the $i$-th point
        \item[] $t_{i}$ - time mark of the $i$-th point
        \item[] $t_{\mathrm{baseline}}$ - right edge of the range used to calculate the baseline (\codet{interval} in the \codet{analyzer::set\tus baseline\tus interval} member-function)
    \end{itemize}
    The second method calculates the baseline as the average value of voltages:
    \begin{equation}
        \mathrm{baseline} = \frac{1}{N}\sum_{t_{i} \in [0, t_{\mathrm{baseline}}]} v_{i},
    \end{equation}
    where $N$ is the number of points in the range from 0 to $t_{\mathrm{baseline}}$ and
    the rest means the same as in Eq.~\eqref{eq:ref:mode}. To use it you should use the \codet{analyzer::BASELINE::AVERAGE} as an argument\\
    Return: & None\\
    Arguments: &
        \begin{tabular}[t]{@{\hspace{0em}}l@{}@{\hspace{1em}}l@{}l}
            \codet{BASELINE method} & - Method identifier for the baseline calculation 
        \end{tabular}\\
    \bottomrule
\end{tabularx}
\vspace{1cm}

\begin{lstlisting}
float get_gate_time() const
\end{lstlisting}
\begin{tabularx}{\textwidth}{rp{11cm}}
    \toprule
    Description: & Returns the time difference between edges of integration range (\textcolor{red}{AC})\\
    Return: & Length of integration range (in $\mu$s)\\
    Arguments: & None\\
    \bottomrule
\end{tabularx}
\vspace{1cm}

\begin{lstlisting}
float get_integral_start() const
\end{lstlisting}
\begin{tabularx}{\textwidth}{rp{11cm}}
    \toprule
    Description: & Returns left edge of integration range (\textcolor{red}{AC})\\
    Return: & left edge of integration range (in $\mu$s)\\
    Arguments: & None\\
    \bottomrule
\end{tabularx}
\vspace{1cm}

\begin{lstlisting}
float get_integral_stop() const
\end{lstlisting}
\begin{tabularx}{\textwidth}{rp{11cm}}
    \toprule
    Description: & Returns right edge of integration range (\textcolor{red}{AC})\\
    Return: & Right edge of integration range (in $\mu$s)\\
    Arguments: & None\\
    \bottomrule
\end{tabularx}
\vspace{1cm}

\begin{lstlisting}
float get_baseline() const
\end{lstlisting}
\begin{tabularx}{\textwidth}{rp{11cm}}
    \toprule
    Description: & Returns the baseline of a signal (\textcolor{red}{AC})\\
    Return: & Baseline level (in mV)\\
    Arguments: & None\\
    \bottomrule
\end{tabularx}
\vspace{1cm}

\begin{lstlisting}
float get_amplitude() const
\end{lstlisting}
\begin{tabularx}{\textwidth}{rp{11cm}}
    \toprule
    Description: & Returns the maximum displacement from the baseline (\textcolor{red}{AC})\\
    Return: & Maximum amplitude (in mV)\\
    Arguments: & None\\
    \bottomrule
\end{tabularx}
\vspace{1cm}

\begin{lstlisting}
float get_amplitude_time() const
\end{lstlisting}
\begin{tabularx}{\textwidth}{rp{11cm}}
    \toprule
    Description: & Returns the time of the maximum displacement from the baseline (\textcolor{red}{AC})\\
    Return: & Time of the maximum point (in $\mu$s)\\
    Arguments: & None\\
    \bottomrule
\end{tabularx}
\vspace{1cm}

\begin{lstlisting}
float get_pkpk() const
\end{lstlisting}
\begin{tabularx}{\textwidth}{rp{11cm}}
    \toprule
    Description: & Returns the voltage difference between the maximum and minimum of a waveform (\textcolor{red}{AC})\\
    Return: & Voltage difference between the extremums (in mV)\\
    Arguments: & None\\
    \bottomrule
\end{tabularx}
\vspace{1cm}

\begin{lstlisting}
float get_integral() const
\end{lstlisting}
\begin{tabularx}{\textwidth}{rp{11cm}}
    \toprule
    Description: & Returns the integral over time in the range specified with the \codet{analyzer::set\tus gate} member-function (\textcolor{red}{AC})\\
    Return: & Integral (in mV$\times$ $\mu$s)\\
    Arguments: & None\\
    \bottomrule
\end{tabularx}
\vspace{1cm}

\begin{lstlisting}
void print() const
\end{lstlisting}
\begin{tabularx}{\textwidth}{rp{11cm}}
    \toprule
    Description: & Prints the result of analysis (\textcolor{red}{AC})\\
    Return: & None\\
    Arguments: & None\\
    \bottomrule
\end{tabularx}
\vspace{1cm}

\newpage
\subsection{class \codet{OwonTreeCreator} (ROOT)}\label{ssec:ref:tree}
\subsubsection*{Description}

\hspace{\parindent} This class is used to create a \codet{TTree} from several binary files

\subsubsection*{\codet{public enum}s}
\begin{lstlisting}
enum SAMPLE { ALL, INCLUDE, EXCLUDE };
\end{lstlisting}
\begin{tabularx}{\textwidth}{rp{11cm}}
    \toprule
    Description: & Should be used to choose what directories to include when constructing
    a tree. See the \codet{OwonTreeCreator::CreateTree} function\\
    Values: & \codet{ALL, INCLUDE, EXCLUDE}\\
    \bottomrule
\end{tabularx}
\vspace{1cm}

\subsubsection*{\codet{public} members}
\begin{lstlisting}
OwonTreeCreator( owon::oscilloscope* osc, const std::string& pathToDataDir, const std::string& pathToTreeFile, const std::treeFileName )
\end{lstlisting}
\begin{tabularx}{\textwidth}{rp{11cm}}
    \toprule
    Description: & Constructor\\
    Arguments: &
        \begin{tabular}[t]{@{\hspace{0em}}l@{}@{\hspace{1em}}l@{}l}
            \codet{owon::oscilloscope* osc} & - pointer to an \codet{oscilloscope} object.                              \\ & Must be of the same model as the one used to record data\\
            \codet{const std::string\& pathToDataDir} & - A valid path to the data directory (see Sec.\ref{ssec:ROOT})\\
            \codet{const std::string\& pathToTreeFile} & - A valid path where the \codet{.root} file will be placed\\
            \codet{const std::string\& treeFileName} & - Name of the \codet{.root} file 
            with the resulting \codet{Tree}s
            \\ & (without an extension)\\
        \end{tabular}\\
    \bottomrule
\end{tabularx}
\vspace{1cm}

\begin{lstlisting}
void SetPathToDataDir( const std::string& pathToDataDir )
\end{lstlisting}
\begin{tabularx}{\textwidth}{rp{11cm}}
    \toprule
    Description: & Should be used to set path to the data directory\\
    Return: & None \\
    Arguments: &
        \begin{tabular}[t]{@{\hspace{0em}}l@{}@{\hspace{1em}}l@{}l}
            \codet{const std::string\& pathToDataDir} & - A valid path to the directory
            which will be used\\
            & to iterate through to create \codet{TTree}s\\
        \end{tabular}\\
    \bottomrule
\end{tabularx}
\vspace{1cm}

\begin{lstlisting}
void SetPathToTreeFile( const std::string& pathToTreeFile )
\end{lstlisting}
\begin{tabularx}{\textwidth}{rp{11cm}}
    \toprule
    Description: & Should be used to set path to the file with the
    resulting \codet{TTree}s\\
    Return: & None \\
    Arguments: &
        \begin{tabular}[t]{@{\hspace{0em}}l@{}@{\hspace{1em}}l@{}l}
            \codet{const std::string\& pathToTreeFile} & - A valid path where the \codet{.root} file with
            \\
            & the resulting \codet{TTree}s will be placed\\
        \end{tabular}\\
    \bottomrule
\end{tabularx}
\vspace{1cm}

\begin{lstlisting}
void SetTreeFileName( const std::string& treeFileName )
\end{lstlisting}
\begin{tabularx}{\textwidth}{rp{11cm}}
    \toprule
    Description: & Should be used to set the name of the resulting \codet{.root} file to which the
    resulting \codet{Tree}s will be written\\
    Return: & None \\
    Arguments: &
        \begin{tabular}[t]{@{\hspace{0em}}l@{}@{\hspace{1em}}l@{}l}
            \codet{const std::string\& treeFileName} & - Name of the file (without an extension)\\
        \end{tabular}\\
    \bottomrule
\end{tabularx}
\vspace{1cm}

\begin{lstlisting}
void SetGate( Double_t beforeTrigger, Double_t afterTrigger )
\end{lstlisting}
\begin{tabularx}{\textwidth}{rp{11cm}}
    \toprule
    Description: & Should be used to set the range of integration \\
    Return: & None \\
    Arguments: &
        \begin{tabular}[t]{@{\hspace{0em}}l@{}@{\hspace{1em}}l@{}l}
            \codet{Double\tus t beforeTrigger} & - pre-trigger time to start integration\\
            \codet{Double\tus t afterTrigger} & - post-trigger time to stop integration\\
        \end{tabular}\\
            & These arguments are in the following connection with those in the \codet{owon::analyzer::set\tus gate} function:\\
            & \codet{integral\tus start} $=$ $t_{\mathrm{trigger}} -$ \codet{beforeTrigger} ,\\ 
            & \codet{integral\tus stop} $=$ \codet{afterTrigger} $+ t_{\mathrm{trigger}}$\\
    \bottomrule
\end{tabularx}
\vspace{1cm}

\begin{lstlisting}
void SetBaselineTime( Double_t baselineTime )
\end{lstlisting}
\begin{tabularx}{\textwidth}{rp{11cm}}
    \toprule
    Description: & Should be used to specify the range for the baseline calculation\\
    Return: & None \\
    Arguments: &
        \begin{tabular}[t]{@{\hspace{0em}}l@{}@{\hspace{1em}}l@{}l}
            \codet{Double\tus t baselineTime} & - time before trigger of the right edge\\
            & of the interval for the baseline calculation\\
        \end{tabular}\\
        & This argument is in the following connection with that in the \codet{owon::analyzer::set\tus baseline\tus time} function:
        \\ & \codet{interval} $=$ $t_{\mathrm{trigger}} - $ \codet{baselineTime}\\
    \bottomrule
\end{tabularx}
\vspace{1cm}

\begin{lstlisting}
void SetActiveChannel( owon::CHANNEL activeChannel )
\end{lstlisting}
\begin{tabularx}{\textwidth}{rp{11cm}}
    \toprule
    Description: & Should be used to set the active channel. It will be used when constructing a \codet{TTree}\\
    Return: & None \\
    Arguments: &
        \begin{tabular}[t]{@{\hspace{0em}}l@{}@{\hspace{1em}}l@{}l}
            \codet{owon::CHANNEL activeChannel} & - The channel to use when constructing a \codet{TTree}\\
        \end{tabular}\\
    \bottomrule
\end{tabularx}
\vspace{1cm}

\begin{lstlisting}
std::string GetPathToDataDir() const 
\end{lstlisting}
\begin{tabularx}{\textwidth}{rp{11cm}}
    \toprule
    Description: & Should be used to get current path to the data directory\\
    Return: & Path to the data directory\\ 
    Arguments: & None\\
    \bottomrule
\end{tabularx}
\vspace{1cm}

\begin{lstlisting}
std::string GetPathToTreeFile() const 
\end{lstlisting}
\begin{tabularx}{\textwidth}{rp{11cm}}
    \toprule
    Description: & Should be used to get current path to the file with the
    resulting \codet{TTree}s\\
    Return: & Path to the file\\ 
    Arguments: & None\\
    \bottomrule
\end{tabularx}
\vspace{1cm}

\begin{lstlisting}
std::string GetTreeFileName() const 
\end{lstlisting}
\begin{tabularx}{\textwidth}{rp{11cm}}
    \toprule
    Description: & Should be used to get the current name of the file which the resulting \codet{TTree}s to write to\\
    Return: & Name of the file (without an extension)\\ 
    Arguments: & None\\
    \bottomrule
\end{tabularx}
\vspace{1cm}

\newpage
%\begin{lstlisting}
%void GetListOfFiles( std::vector< std::string >& fileNames,
%                     bool fullPath = true,
%                     std::string pathToParentDir = "" ) const
%\end{lstlisting}
%\begin{tabularx}{\textwidth}{rp{11cm}}
%    \toprule
%    Description: & \\
%    Return: & \\ 
%    Arguments: &
%        \begin{tabular}[t]{@{\hspace{0em}}l@{}@{\hspace{1em}}l@{}l}
%            \codet{std::vector< std::string >\& fileNames} & -\\
%            \codet{bool fullPath} & -\\
%            \codet{std::string pathToParentDir} & -\\
%        \end{tabular}\\
%    \bottomrule
%\end{tabularx}
%\vspace{1cm}

\begin{lstlisting}
void CreateTree( SAMPLE mode = ALL, const std::string& target = "" )
\end{lstlisting}
\begin{tabularx}{\textwidth}{rp{11cm}}
    \toprule
    Description: & Create a \codet{TTree} from binaries\\
    Return: & None \\ 
    Arguments: &
        \begin{tabular}[t]{@{\hspace{0em}}l@{}@{\hspace{1em}}l@{}l}
            \codet{SAMPLE mode} & - \codet{OwonTreeCreator::SAMPLE::ALL, OwonTreeCreator::SAMPLE::INCLUDE}\\ & or \codet{OwonTreeCreator::SAMPLE::EXCLUDE}\\
            \codet{const std::string\& target} & - target directory name which is used to
            create a \codet{TTree}:\\
            & if the first argument \codet{mode} is \codet{OwonTreeCreator::SAMPLE::ALL} then this\\
            & argument is ignored and all the subdirectories will be used.\\
            & If \codet{mode} is \codet{OwonTreeCreator::SAMPLE::INCLUDE} then only the directory\\
            & named \codet{target} will be used.\\
            & If \codet{mode} is \codet{OwonTreeCreator::SAMPLE::EXCLUDE} then all the directories\\
            & will be used except the one named \codet{target}\\
        \end{tabular}\\
    \bottomrule
\end{tabularx}
\vspace{1cm}
